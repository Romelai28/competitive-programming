%"The PDF file may contain up to 25 pages of reference material, single-sided, letter or A4 size, with text and illustrations readable by a person with correctable eyesight without magnification from a distance of 1/2 meter."
\input{preamble.tex}
\begin{document}

\def\title{Universidad de Buenos Aires - FCEN - Sale con dp y fritas}
.\\[0.2cm]
\centering{\LARGE\textbf{Sale con dp y fritas}} \\[0.5cm]
\centering{\includegraphics[width=5.5cm]{img/fritas.png}}
\tableofcontents\newpage

\section{Template}
\cppfile{other/template.cpp}

\section{Data structures}
\subsection{Disjoint Set}
\cppfile{data_structures/disjointSet.cpp}
\subsection{MinHeap}
\cppfile{data_structures/minHeap.cpp}

\section{Grafos}
\subsection{BFS}
\cppfile{graph/bfs.cpp}
\subsection{DFS}
\cppfile{graph/dfs.cpp}
\subsection{Bellman-Ford}
\cppfile{graph/bellman_ford.cpp}
\subsection{Dijkstra}
\cppfile{graph/dijkstra.cpp}
\subsection{Floyd-Warshall}
\cppfile{graph/floyd_warshall.cpp}
\subsection{Dantzig}
\cppfile{graph/dantzig.cpp}
\subsection{Kruskal}
\cppfile{graph/kruskal.cpp}
\subsection{Toposort}
\cppfile{graph/toposort.cpp}

\section{Searching}
\subsection{Binary search}
\cppfile{searching/binary_search.cpp}
\subsection{Integer ternary search}
\cppfile{searching/integer_ternary_search.cpp}
\subsection{Ternary search}
\cppfile{searching/ternary_search.cpp}
\subsection{Intervals}
\cppfile{searching/intervals.cpp}

\section{Matematicas}
\subsection{Integrador Numerico Simpson}
\cppfile{math/simpsons_rule.cpp}
\subsection{Euclides extendido}
\cppfile{math/extended_euclides.cpp}
\subsection{Ecuaciones diofanticas lineales}
\cppfile{math/diophantine_ecuations.cpp}
\subsection{GCD - Maximo comun divisor}
\cppfile{math/greatest_common_divisor.cpp}
\subsection{LCM - Minimo comun multiplo}
\cppfile{math/lowest_common _multiple.cpp}
\subsection{Criba de Eratostenes}
\cppfile{math/sieve_of_eratosthenes.cpp}
\subsection{Fibonacci mod m}
\cppfile{math/fibo_mod.cpp}
\subsection{Teorema Chino del Resto}
\cppfile{math/chinese_remainder_theorem.cpp}

\section{Other}
\subsection{Enumerar abecedario}
\cppfile{other/enumerar_abecedario.cpp}
\subsection{Funciones utiles}
\cppfile{other/funciones_utiles.cpp}


%% COMPLEJIDADES:
Aproximación del mayor número n de datos que pueden procesarse para cada una de las complejidades algoritmicas. Tomar esta tabla solo como referencia.

\begin{tabbing}
\textbf{Complexity}\hspace{4cm} \=  \textbf{n}\hspace{3cm}   \\ 
$O(n!)$ \> 11\\ 
$O(n^{5})$ \> 50\\ 
$O(2^{n}*n^{2})$ \> 18\\ 
$O(2^{n}*n)$ \> 22\\ 
$O(n^{4})$ \> 100\\ 
$O(n^{3})$ \> 500\\ 
$O(n^{2}\log_{2}n)$ \> 1.000\\ 
$O(n^{2})$ \> 10.000\\ 
$O(n\log_{2}n)$ \> $10^{6}$\\ 
$O(n)$ \> $10^{8}$\\ 
$O(\sqrt{n})$ \> $10^{16}$\\ 
$O(\log_{2}n)$ \> -\\ 
$O(1)$ \> -\\ 
\end{tabbing}


\end{document}